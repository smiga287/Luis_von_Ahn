\documentclass{article}
\usepackage[utf8]{inputenc}
\usepackage[T1]{fontenc} 
\newcommand\dJ{{\fontencoding{T1}\selectfont\dj}}\newcommand\Dj{{\fontencoding{T1}\selectfont\DJ}}

\title{Luis von Ahn}
\author{petrickovicm99 }
\date{November 2019}

\begin{document}

\maketitle

\section{Biografija}

\section{Rad}
Godine 2000, u saradnji sa Manuelom Blumom, bavio se pionirskim radom na \textit{CAPTCHA} sistemu - kompjuterski generisanim testovima koje ljudi mogu jednostavno da urade, ali koje raCunari nisu u stanju da reSe. Ovaj sistem koriste internet stranice da bi spreCile automatizovane programe, botove, da poCine zloupotrebu velikih razmera, kao Sto je automatsko registrovanje mnogobrojnih naloga ili kupovina velikog broja ulaznica koje bi preprodavci kasnije preprodavali. \textit{CAPTCHA} je proslavila Fon Ana pred Sirokim auditorijumom zbog toga Sto su o njoj Clanke objavili \textit{New York Times} i \textit{USA Today} i Sto je propraCena na kanalima \textit{Discovery Channel}, \textit{NOVA scienceNOW} i ostalim popularnim medijima.

Fon Anova doktorska teza, zavrSena 2005. godine, bio je prvi struCni rad koji je koristio termin \textbf{/human computation/}, koga je smislio Fon An sa Zeljom da opiSe metode koje kombinuju ljudsku inteligenciju i raCunare da bi reSile probleme koji u suprotnom ne bi mogli da se reSe. Fon Anova doktorska teza je takoDJe prvi struCni rad na temu \textbf{/Igre sa poentom/} (\textit{Games With A Purpose - GWAP}). Igre sa poentom su igre koje, kao propratni efekat, izvode korisne proraCune. Najpoznatiji primer je \textit{ESP Game}, onlajn igra tokom koje se nasumicnom paru igraca bez moguCnosti komunikacije istovremeno prikazuje ista slika koju oni treba da opiSu u ogranicenom vremenu, za Sta dobijaju poene. Na osnovu ove igre dobija se precizan opis slike koji moZe biti uspeSno upotrebljen u bazi podataka za poboljSavanje tehnologije pretrage slika. Gugl je licencirao \textit{ESP Game} u obliku Gugl obeleZivaca slika (engl. \textit{Google Image Labeler}) i koristi se da poboljSa preciznost Gugl pretraZivaCa slika (engl. \textit{Google Image Search}). Fon Anova igra donela mu je paZnju popularnih medija. Njegova teza osvojila je nagradu najbolje doktorske disertacije Skole raCunarskih nauka univerziteta Karnegi Melon. Jula 2006, Fon An je odrZao govor u Guglu o \textit{/Human Computation/} (tj. \textit{crowdsourcing}) koji je imao preko milion gledalaca.


Fon An je 2007. godine izumeo \textit{reCAPTCHA}, novi oblik \textit{CAPTCHA} koji takode pomaZe pri digitalizaciji knjiga. Slike reCi koje koristi \textit{reCAPTCHA} koje dolaze direktno iz starih knjiga nakon Sto su digitalizovane; koriste se reCi koje optiCko prepoznavanje znakova ne bi moglo da prepozna i Salju se ljudima putem mreZe kako bi se identifikovali. \textit{reCAPTCHA} se trenutno koristi na preko 100.000 veb-sajtova i prevodi preko 40 miliona reci dnevno.

Godine 2011. dodeljena mu je nagrada A. Niko Habermana, predsedavajuCeg za razvoj u raCunarskim naukama, koja se dodeljuje svake treCe godine obeCavajuCem mladom Clanu fakulteta u Skoli raCunarskih nauka.

Fon An 2018. godine osvojio je \textit{Lemelson-MIT} nagradu za /njegovu posveCenost poboljSanju sveta kroz tehnologiju/.

Od 2014. Fon An je generalni direktor Duolinga, platforme za uCenje jezika.
\end{document}