\documentclass{article}
\usepackage[utf8]{inputenc}

%\newcommand\dJ{{\fontencoding{T1}\selectfont\dj}}\newcommand\Dj{{\fontencoding{T1}\selectfont\DJ}}

\title{Luis von Ahn}
\author{petrickovicm99 }
\date{November 2019}

\begin{document}

\maketitle

\section{Biografija}
Fon An je roDJen i odrastao u Gvatemali. On je nemačko-gvatemalskog porekla. 
Osnovne studije iz matematike (\textit{summa cum laude}) je završio na Djuk univerzitetu 2000. godine, a kasnije doktorat iz računarstva na Karnegi Melon univerzitetu 2005.
Fon An je 2006. godine postao fakultetski član Karnegi Melon škole Računarstva na univerzitetu Karnegi Melon.

\section{Rad}
Kao profesor, njegova istraživanja uključuju \textit{CAPTCHA} i \textit{human computation}, čime je stekao svoju svetsku prepoznatljivost i brojne nagrade. Nagrađen je \textit{MacArthur Fellowship 2006}, \textit{David and Lucile Packard Foundation Fellowship 2009, Sloan Fellowship 2009, Microsoft New Faculty Fellowship 2007. i Presidental Early Career Award for Scientists and Engineers 2012}. TakoDJe je bio imenovan jednim od 50 najvećih umova u nauci od strane \textit{Discover}, i dospeo je do mnogih uglednih listi kao što su \textit{Briliant 10} od \textit{Popular Science}, \textit{50 Most Influential People in Technology} od \textit{Silicon.com}, \textit{TR35: Young Innovators Under 35} od \textit{MIT Technology Review i 100 Most Innovative People in Business} od \textit{Fast Company}. 
\textit{Siglo Veintiuno}, jedna od najvećih novina u Gvatemali, izabrala ga je za osobu godine 2009. \textit{Foreign Policy Magazine} na španskom ga je 2011. imenovao za najuticajnijeg intelektualca Latinske Amerike i Španije. 
Fon An je svoja istraživanja započeo na polju kriptografije. Zajedno sa Nikolasom Huperom i Džonom Langfordom, prvi je obezbedio precizne definicije steganografije i dokazao da je steganografija sa ličnim ključem moguća.

\end{document}

