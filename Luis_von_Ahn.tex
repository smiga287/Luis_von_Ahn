\documentclass[titlepage, 12pt]{article}
\usepackage[utf8]{inputenc}
\usepackage{multirow}
\usepackage{graphicx}
\usepackage[english,serbian]{babel}

\title{Luis fon An\\ \small{Seminarski rad u okviru kursa\\Tehničko i naučno pisanje\\ Matematički fakultet}}
\author{Aleksandar Šmigić \\ mi19028@alas.matf.bg.ac.rs \and Andrijana Milić \\ mi18186@alas.matf.bg.ac.rs \and Miloš Petričković \\ mi18055@alas.matf.bg.ac.rs \and Nikola Milutinović \\ mi18202@alas.matf.bg.ac.rs}
\date{Novembar 2019}

\begin{document}

\maketitle

\tableofcontents
\newpage
\section{Uvod}

{\renewcommand{\arraystretch}{1.2}}
\begin{tabular}{|c|c|c|}
\hline
\multicolumn{2}{|c|}{\Large{Luis fon An}}\\[4px]
\hline
\multicolumn{2}{|c|}{\includegraphics[width=320px,height=216px]{Luis_von_Ahn.jpg}}\\

\hline
Datum rođenja & 19.8.1955. \\
\hline
Mesto rođenja & Gvatemala, Gvatemala\\
\hline
Prebivalište & Berkli(Kalifornija)\\
\hline
\multirow{2}{*}{Institucije}&Univerzitet Karnegi Melon\\
& Univerzitet Djuk \\
\hline 
\multirow{3}{*}{Poznat po} & CAPTCHA \\
& reCAPTCHA \\ & Duolingo \\
\hline
\multirow{2}{*}{Nagrade} & MacArthur Fellowship (2006) \\
 & TR35 (2007)
\\
\hline
\end{tabular}

\vspace{20px}

Luis fon An, rođen 19. avgusta 1978. je gvatemalski preduzetnik i profesor konsultant na odseku za računarstvo na univerzitetu Karnegi Melon u Pitsburgu u Pensilvaniji.[1] Poznat je kao jedan od začetnika crowdsourcing-a. On je osnivač kompanije reKapča (engl. reCAPTCHA), koja je prodata Guglu 2009. godine, [2] i jedan od osnivača i glavni izvršni direktor Duolinga, popularne platforme za učenje jezika. 

\section{Biografija}
Fon An je roDJen i odrastao u Gvatemali. On je nemačko-gvatemalskog porekla. 
Osnovne studije iz matematike (\textit{summa cum laude}) je završio na Djuk univerzitetu 2000. godine, a kasnije doktorat iz računarstva na Karnegi Melon univerzitetu 2005.
Fon An je 2006. godine postao fakultetski član Karnegi Melon škole Računarstva na univerzitetu Karnegi Melon.

\section{Rad}
Kao profesor, njegova istraživanja uključuju \textit{CAPTCHA} i \textit{human computation}, čime je stekao svoju svetsku prepoznatljivost i brojne nagrade. Nagrađen je \textit{MacArthur Fellowship 2006}, \textit{David and Lucile Packard Foundation Fellowship 2009, Sloan Fellowship 2009, Microsoft New Faculty Fellowship 2007. i Presidental Early Career Award for Scientists and Engineers 2012}. TakoDJe je bio imenovan jednim od 50 najvećih umova u nauci od strane \textit{Discover}, i dospeo je do mnogih uglednih listi kao što su \textit{Briliant 10} od \textit{Popular Science}, \textit{50 Most Influential People in Technology} od \textit{Silicon.com}, \textit{TR35: Young Innovators Under 35} od \textit{MIT Technology Review i 100 Most Innovative People in Business} od \textit{Fast Company}. 
\textit{Siglo Veintiuno}, jedna od najvećih novina u Gvatemali, izabrala ga je za osobu godine 2009. \textit{Foreign Policy Magazine} na španskom ga je 2011. imenovao za najuticajnijeg intelektualca Latinske Amerike i Španije. 
Fon An je svoja istraživanja započeo na polju kriptografije. Zajedno sa Nikolasom Huperom i Džonom Langfordom, prvi je obezbedio precizne definicije steganografije i dokazao da je steganografija sa ličnim ključem moguća.

Godine 2000, u saradnji sa Manuelom Blumom, bavio se pionirskim radom na \textit{CAPTCHA} sistemu - kompjuterski generisanim testovima koje ljudi mogu jednostavno da urade, ali koje raCunari nisu u stanju da reSe. Ovaj sistem koriste internet stranice da bi spreCile automatizovane programe, botove, da poCine zloupotrebu velikih razmera, kao Sto je automatsko registrovanje mnogobrojnih naloga ili kupovina velikog broja ulaznica koje bi preprodavci kasnije preprodavali. \textit{CAPTCHA} je proslavila Fon Ana pred Sirokim auditorijumom zbog toga Sto su o njoj Clanke objavili \textit{New York Times} i \textit{USA Today} i Sto je propraCena na kanalima \textit{Discovery Channel}, \textit{NOVA scienceNOW} i ostalim popularnim medijima.

Fon Anova doktorska teza, zavrSena 2005. godine, bio je prvi struCni rad koji je koristio termin \textbf{/human computation/}, koga je smislio Fon An sa Zeljom da opiSe metode koje kombinuju ljudsku inteligenciju i raCunare da bi reSile probleme koji u suprotnom ne bi mogli da se reSe. Fon Anova doktorska teza je takoDJe prvi struCni rad na temu \textbf{/Igre sa poentom/} (\textit{Games With A Purpose - GWAP}). Igre sa poentom su igre koje, kao propratni efekat, izvode korisne proraCune. Najpoznatiji primer je \textit{ESP Game}, onlajn igra tokom koje se nasumicnom paru igraca bez moguCnosti komunikacije istovremeno prikazuje ista slika koju oni treba da opiSu u ogranicenom vremenu, za Sta dobijaju poene. Na osnovu ove igre dobija se precizan opis slike koji moZe biti uspeSno upotrebljen u bazi podataka za poboljSavanje tehnologije pretrage slika. Gugl je licencirao \textit{ESP Game} u obliku Gugl obeleZivaca slika (engl. \textit{Google Image Labeler}) i koristi se da poboljSa preciznost Gugl pretraZivaCa slika (engl. \textit{Google Image Search}). Fon Anova igra donela mu je paZnju popularnih medija. Njegova teza osvojila je nagradu najbolje doktorske disertacije Skole raCunarskih nauka univerziteta Karnegi Melon. Jula 2006, Fon An je odrZao govor u Guglu o \textit{/Human Computation/} (tj. \textit{crowdsourcing}) koji je imao preko milion gledalaca.


Fon An je 2007. godine izumeo \textit{reCAPTCHA}, novi oblik \textit{CAPTCHA} koji takode pomaZe pri digitalizaciji knjiga. Slike reCi koje koristi \textit{reCAPTCHA} koje dolaze direktno iz starih knjiga nakon Sto su digitalizovane; koriste se reCi koje optiCko prepoznavanje znakova ne bi moglo da prepozna i Salju se ljudima putem mreZe kako bi se identifikovali. \textit{reCAPTCHA} se trenutno koristi na preko 100.000 veb-sajtova i prevodi preko 40 miliona reci dnevno.

Godine 2011. dodeljena mu je nagrada A. Niko Habermana, predsedavajuCeg za razvoj u raCunarskim naukama, koja se dodeljuje svake treCe godine obeCavajuCem mladom Clanu fakulteta u Skoli raCunarskih nauka.

Fon An 2018. godine osvojio je \textit{Lemelson-MIT} nagradu za /njegovu posveCenost poboljSanju sveta kroz tehnologiju/.

Od 2014. Fon An je generalni direktor Duolinga, platforme za uCenje jezika.


\section{Učenje}

Fon An je koristio mnoge neobične tehnike u svojoj nastavi koje su mu donele brojne nagrade Univerziteta u Karnegi Melonu. Jeseni 2008. godine započeo je novi kurs na Karnegi Melonu sa nazivom „Nauka mreža” (engl. Science of the Web). Kombinacija teorije grafova i društvenih nauka, kurs obrađuje teme od teorije mreža i teorije igara do teorija aukcije.

\begin{thebibliography}{9}
\bibitem{ref1}
    \textit{„Luis von Ahn”.}
\bibitem{ref2}
    \textit{„Teaching computers to read: Google acquires reCAPTCHA”. Google Official Blog.}
\bibitem{ref3}
    \textit{„Duke Ugrad Alum Profile: Luis von Ahn”.}
\bibitem{ref4}
    \textit{https://www.cs.cmu.edu/~biglou/LuisvonAhnCV.pdf}
\bibitem{ref5}
    \textit{Federoff, Stacey. „Duolingo CEO Luis von Ahn wins Lemelson prize from MIT”.}
\bibitem{ref6} 
    \textit{Robert J. Simmons. „Profile Luis von Ahn: ReCaptcha, games with a purpose”. XRDS: Crossroads, the ACM Magazine for Students.}
\bibitem{ref7}
    \textit{„MacArthur Fellows”.}
\bibitem{ref8}
    \textit{„Congratulations, Luis von Ahn”. Google Official Blog.}
\bibitem{ref9}
    \textit{„President Obama Honors Outstanding Early-Career Scientists”. The White House.}
\bibitem{ref10}
    \textit{„Los nuevos rostros del pensiamento iberoamericano”. FP.}
\bibitem{ref11} 
    \textit{„Luis von Ahn”.}
\bibitem{ref12} 
    \textit{Von Ahn, Luis; Blum, Manuel; Hopper, Nicholas J.; Langford, John). "CAPTCHA: Using Hard AI Problems for Security". Proceedings of the International Conference on the Theory and Applications of Cryptographic Techniques}
\bibitem{ref13}
    \textit{Von Ahn, L.; Blum, M.; Langford, J. „Telling humans and computers apart automatically”. Communications of the ACM.}
\bibitem{ref14}
    \textit{Von Ahn, L.; Dabbish, L. „Labeling images with a computer game”. Proceedings of the conference on Human factors in computing systems}
\bibitem{ref15}
    \textit{Von Ahn, L. „Games with a Purpose”.}
\bibitem{ref16} Google Tech Talk on human computation by Luis von Ahn.
\bibitem{ref17}
    \textit{„reCAPTCHA (a.k.a. Those Infernal Squiggly Words) Almost Done Digitizing the New York Times Archive”. Blog.newsweek.com.}
\bibitem{ref18}
    \textit{Habermmann Chair announcement. News.cs.cmu.edu}
\bibitem{ref19}
    \textit{Federoff, Stacey. „Duolingo CEO Luis von Ahn wins Lemelson prize from MIT”. TribLIVE.com.}
\bibitem{ref20}
    \textit{Duolingo.com}
\bibitem{ref21}
    \textit{CMU Faculty Awards.}
\bibitem{ref22}
    \textit{„Science of the Web”. Andrew.cmu.edu.}
\end{thebibliography}

\end{document}
