\documentclass[titlepage, 12pt]{article}
\usepackage[utf8]{inputenc}


\usepackage[english,serbian]{babel}

\title{Luis fon An\\ \small{Seminarski rad u okviru kursa\\Tehničko i naučno pisanje\\ Matematički fakultet}}
\author{Aleksandar Šmigić \\ mi19028@alas.matf.bg.ac.rs \and Andrijana Milić \\ mi18186@alas.matf.bg.ac.rs \and Miloš Petričković \\ mi18055@alas.matf.bg.ac.rs \and Nikola Milutinović \\ mi18202@alas.matf.bg.ac.rs}
\date{Novembar 2019}

\begin{document}

\maketitle

\tableofcontents
\newpage
\section{Uvod}
Luis fon An, rođen 19. avgusta 1978. je gvatemalski preduzetnik i profesor konsultant na odseku za računarstvo na univerzitetu Karnegi Melon u Pitsburgu u Pensilvaniji.[1] Poznat je kao jedan od začetnika crowdsourcing-a. On je osnivač kompanije reKapča (engl. reCAPTCHA), koja je prodata Guglu 2009. godine, [2] i jedan od osnivača i glavni izvršni direktor Duolinga, popularne platforme za učenje jezika. 

\end{document}
